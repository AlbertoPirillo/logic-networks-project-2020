\documentclass[a4paper, 12pt]{article}
\usepackage[utf8]{inputenc}
\usepackage[italian]{babel}
\usepackage{imakeidx}
\usepackage{graphicx}
\usepackage{titlesec}

% Custom margins
\usepackage[a4paper, inner=1.5cm, outer=2.7cm, top=3cm, bottom=2cm, bindingoffset=1.2cm]{geometry}

% List spacing
\usepackage{enumitem}
\setlist{topsep=2pt, itemsep=2pt, partopsep=2pt, parsep=2pt}

% Header and footer
\usepackage{fancyhdr}
\pagestyle{fancy}
\fancyhf{}
\lhead{Prova finale - Reti Logiche}
\rhead{Alberto Pirillo}
\cfoot{\thepage}

% Tables
\usepackage{tabu}
\usepackage{caption} 
\captionsetup[table]{skip=2pt}

% Title 
\title{Prova finale - Reti Logiche}
\author{Alberto Pirillo - 10667220 \\ Prof. Gianluca Palermo}
\date{Anno Accademico 2020/2021}

\makeindex

\begin{document}

\maketitle
\tableofcontents
\pagebreak

\section{Introduzione}
\subsection{Scopo del progetto}
\subsection{Specifiche generali}
\subsection{Memoria e dati}

\section{Architettura}
\subsection{Interfaccia del componente}
\subsection{Scelte progettuali}
\subsection{Descrizione degli stati}

\section{Risultati sperimentali}
\subsection{Report di sintesi}
\subsection{Simulazioni}
\subsubsection{Pre-sintesi}
\subsubsection{Post-sintesi}

\section{Conclusioni}


\end{document}
